\documentclass[a4paper, 11pt]{article}
\setlength{\topmargin}{-0.5in}
\setlength{\textheight}{9.5in}
\setlength{\oddsidemargin}{-.1in}
\setlength{\textwidth}{6.5in}

\usepackage{multirow}
\usepackage{float}
\usepackage{array}
\usepackage[document]{ragged2e}

\usepackage{datetime}

\newdateformat{mydate}{\monthname[\THEMONTH] \THEYEAR}

\newcolumntype{L}{>{\centering\arraybackslash}m{3cm}}

\usepackage{graphicx}
\graphicspath{{images/}}

\begin{document}


\LARGE\title{User Modeling in search for People with Autism}

\LARGE\author{Author: \textbf{Esha Massand}, Supervisor: \textbf{Keith Mannock}\\
\\
Birkbeck, University of London\\Department of Computer Science and Information Systems\\\\\\Project report submitted in partial fulfillment of the requirement for the MSc in Computer Science\date{\mydate\today}
\\\
}

\normalsize


\maketitle


\section*{Abstract}
\begin{justify}
This project report presents the development of a prototype web application to assist users with Autism when they search the web. The developed system models user interactions with the search process into a user profile by integrating insights from the core features of autism into the model. The user profile is applied to a synthesis of three leading search engines, and the entire system is integrated with an infra-red user interface component to assist users with Autism during search.\\
\end{justify}


\begin{justify}
This project is substantially the result of my own work, expressed in my own words, except where explicitly indicated in the text. I give my permission for it to be submitted to a Plagiarism Detection Service. This proposal may be freely copied and distributed provided the source is explicitly acknowledged.
\end{justify}

\begin{verbatim}














\end{verbatim}

\begin{center}
word count (proposal text only) : 3294 words.
\end{center}

\clearpage
\tableofcontents
\clearpage

\section*{Abbreviations}
\begin{tabular}{l l }
API & Application Programming Interface\\
ASD & Autism Spectrum Disorder\\
DSM & Diagnostic and Statistical Manual\\
GCS & Google Custom Search\\
HCI & Human Computer Interaction\\
HTTP & Hypertext Transfer Protocol\\
JSON & JavaScript Object Notation\\
IDE & Integrated Development Environment\\
KWIC & Key Word In Context\\
LEAP & LEAP Motion Controller\\
REST & Representational state transfer\\
RIFT & Oculus Rift Virtual Reality Head Mounted Display\\
TDD & Test Driven Development\\
UI & User Interface\\
UX & User Experience\\
VR & Virtual Reality\\
\end{tabular}

\section*{Definitions}

\begin{tabular}{l p{15cm}  }
Autism & Autism is amongst the most common neurodevelopmental condition and it is currently estimated that 1/68 children meet criteria for Autism Spectrum \cite{CDC}. Autism is five times more common amongst boys than girls (1/42 boys, and 1/189 girls). According to the Diagnostic and Statistical Manual (2013), Autism is characterized by persistent and early deficits in reciprocal social interaction and repetitive behaviours. Individuals vary from high functioning to low functioning (along a spectrum), with behaviours emerging around 2 to 3 years of age.
\end{tabular}
\clearpage

\section{Introduction}\label{intro}

This project report presents the development and evaluation of a web-browser based application hereafter referred to as Jellibean. Jellibeans are a rainbow of colours, different sizes and shades, and the name represents the difference in style of processing of individuals with ASD. 

\section{}

\section {Aims and Objectives}
The goal of the project is to build a prototype search tool that assists users with Autism search and navigate the web. To acheive this goal, the aims of this work are:
\begin{enumerate}
\item{
To synthesise the search results from three search engines, Google, Bing and Yahoo. For search results returned by Google, the Custom Search API will be used in line with the Google terms of service (as `screen scraping', or copying the data directly from the website is prohibited). It is a RESTful api with a single method called list. The API method used was GET, and the response data is returned as a JSON type. The response consists of (1) the actual search result, (2) metadata for search like number of  results, alternative search queries, and (3) custom search engine metadata. The data model depends on \cite{opensearch}.

For Bing and Yahoo search results, JSoup (a Java HTML parser) will be used to identify the links from the resulting query. The JSoup HTML parser was considered more effient for retrieving search results, as it reduced the number of lines of code required to complete the task. 

Jsoup has its advantages over html parsing. It contains a class representing a list of nodes, `Elements', which implements Iterable to iterate over a list in an enhanced for loop.

The resulting links are written to text file which stores the links in a text file in the projects source directory.}

\item{
To build a stereotyped user model (a user model that will infer characteristics about the user from their diagnostic information) for a person with ASD. Users will have to register will Google+, and report their diagnostic information in the aboutMe section of their profile. Using the Google+ API, Jellibeans will connect to Google+ and ask the user to signin/ agree to the web application accessing their data. JavaScript will be used to parse the text in the aboutMe section to check if the user has identified as a persn with autism, aspergers, ASD or not. The result is returned via the console (inspect) command.
}

\item{
I will test and evaluate the system. Testing will involve assessing the reliability and robustness of Jellibeans; the ease of its interaction; boundary conditions; ease of use; does it fullfil the aims of the project.\n 

Evaluation of the system will include comparisons to existing search engines; assessing how this idea can be implemented to tailor an existing systems; assessing how well the system does compared to existing systems on a set of criteria that are only relevant to the user group in question. Evaluation will also include quantitative metrics such as Recall, Precision, and False Negative/Positive rates.
}

\item {Apache Solr}



\end{enumerate}

This work has successfully completed aims XXX

\section{User modelling research and methodolody}
This section describes the process of how a user model of autism was built. 
\subsection{User requirements}
\subsection{User behaviours}
\subsection{Design of user model}

\section{Development and Implementation of Jellibean}
\subsection{API's and Development Tools}
\subsection{Performance}

\section{Integrating Jellibean with Motion Controller Hardware}
Current Project’s Hardware Selection Process and Important Design Issues:
\begin{enumerate}
\item{Good timing correlates to a good meaning and User Experience.}
\item{The leap has options to ‘poll’ frames at a constant rate (to keep timing of movement accurate) which is important.}
\item{Cognitive ‘lag’ time. Each of our senses operates with a different lag time. Hearing has the fastest sense-to-cognition/understanding time, and surprisingly sight -- the slowest. If the devices interferes with the processing of the sense, it will confuse the combinatorial configuration of the senses, leading to misunderstandings in the meaning and a worse user experience.}
\item{Volume is important because this is a tool to be used with individuals with ASD, the device must have a low `volume’, i.e., the sensory experience cannot be overwhelming.}
\item{Load, by this I mean `cognitive' load is most desirable when not high. We do not want the device to be overwhelming in terms of it’s cognitive load.}
\item{Within the selection process, I did not just consider the physical design of the device, but also the way in which the devices manifests actions into behaviours. That is, how does the user engage behaviourally within the environment using the device? What about the physical sensation and its path towards a behavioural or emotional response? For example, can we program there to be an activity followed by a reward to reinforce the activity.}
\end{enumerate}	

\section {Critical Review of the Leap Motion Controller}
Advantages
\begin{enumerate}
\item{Impressive}
\item{Uses infrared to embed the users (phantom) hand™ on the screen}
\item{New technology and novel to bring to laptops}
\item{East to set up}
\item{Has built in gestures and navigation tools}
\item{Can work in pretty dimly lit environments (but not all)}
\item{It is sophisticated (sometimes the polling frequency lets it down)}
\item{Picks up an impressive distance along the z axis}
\item{Offers a recalibration process if the controller is persistently jumpy, or there are discontinuities in the tracking data, if there are aberrations in the tracking data that occur in certain areas of the field of view, or poor tracking range. This can be done using the shiny surface such as the computer screen or mirror.}
\end{enumerate}	
Disadvantages
\begin{enumerate}
\item{Misses small hand movements}
\item{The range that it will detect is 150 degree angle along the y axis, this is reasonable but not always idea when gestures/hand movements are large.}
\item{Some parts of the screen the hands to not ‘reach’, i.e., bottom left /right of screen are sometimes hard to reach.}
\item{Loses the hand, stops working/sensing the hands, even when the controller stays on.}
\item{Often misses frames, so the user makes larger hand movements and then overshoots (when the LEAP catches up)}
\item{Built-in controls are not ideal}
\item{Lighting works best when the hand is seen in silhouette fashion by the controller. }
\end{enumerate}

\section{Jellibean Showcase}

\section{Conclusion}

\clearpage
\begin{thebibliography}{100}
\bibitem {opensearch}OpenSearch, http://www.opensearch.org/Specifications/OpenSearch/1.1, Retrieved 30 July 2015.
\end{thebibliography}
\end{document}